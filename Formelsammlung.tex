\documentclass[a4paper]{article}

\usepackage{amsmath}
\usepackage{amssymb}
\usepackage{amsthm}
\usepackage[ngerman]{babel}
\usepackage{enumitem}
\usepackage[utf8]{inputenc}
\usepackage[T1]{fontenc}
\usepackage[left=25mm,right=25mm,top=25mm,bottom=25mm]{geometry}
\usepackage[colorlinks=true,urlcolor=blue,linkcolor=black]{hyperref}

\begin{document}
\setlength{\parindent}{0mm}

\title{\textbf{MATHEMATIK}}
\author{Der Robert}
\maketitle

\tableofcontents

\newpage
\section{Funktionen}

\subsection{Polynom $n$-ten Grades}
Eine Funktion ist ein Polynom $n$-ten Grades,
wenn sie folgende Form hat:
\[
	f(x) = p_1 \cdot x^n + p_2 \cdot x^{n - 1} + p_3 \cdot x^{n - 2} + ... + p_{n - 2} \cdot x^2 + p_{n - 1} \cdot x + p_n ;
\]
Hierbei stehen
$p_1 ... p_n \in \mathbb{R} $
f\"ur beliebige Faktoren zum jeweiligen Polynom.

\subsection{Spezialf\"alle}

\subsubsection{Polynom des Grades $0$}
Bei einem Polynom des Grades $0$ handelt es sich um eine Gerade,
die parallel zur $x$-Achse verl\"auft.
Sie lautet:
\[
	f(x) = c ;
\]
Wobei $c \in \mathbb{R} $ ist,
und nicht $0$ sein sollte.
Der Buchstabe $c$ als Variable leitet sich davon ab,
dass es sich um eine Konstante handelt,
auf Englisch auch \textit{constant},
von lateinisch \textit{c\=onst\=are} = \textit{feststehen},
bzw. \textit{c\=onst\=ans} = \textit{feststehend}.

\subsubsection{Polynom ersten Grades}
Bei einem Polynom ersten Grades handelt es sich lediglich um eine lineare Funktion:
\[
f(x) = m \cdot x + t ;
\]
Sie hat die Steigung $m$ und den Nullpunkt $ -t $.
Ist $m$ positiv,
ist die Funktion monoton steigend;
ist $m$ negativ,
ist die Funktion monoton fallend.

\subsubsection{Polynom zweiten Grades}
Eine Funktion zweiten Grades ist eine quadratische Funktion.
Ihr Funktionsgraph ist eine Parabel.
\[
	f(x) = a \cdot x^2 + b \cdot x + c ;
\]
Die Nullpunkte lassen sich \"uber die sog. Mitternachtsformel bestimmen:
\[
	x_{1,2} = \frac{-b \pm \sqrt{b^2 - 4 \cdot a \cdot c}}{2 \cdot a} ;
\]
Man kann anhand der Diskriminante auch bestimmen,
wie viele Nullstellen eine quadratische Funktion hat.
Die Diskriminante ist der Radikand der Mitternachtsformel,
also $ b^2 - 4 \cdot a \cdot c $.
Jetzt gibt es folgende M\"oglichkeiten:
\begin{itemize}[nosep]
	\item $ b^2 - 4 \cdot a \cdot c < 0 $: Die Funktion hat keinen Nullpunkt.
	\item $ b^2 - 4 \cdot a \cdot c = 0 $: Die Funktion hat genau einen Nullpunkt, n\"amlich $ -\frac{b}{2 \cdot a} $
	\item $ b^2 - 4 \cdot a \cdot c > 0 $: Die Funktion hat zwei Nullpunkte.
\end{itemize}

\subsubsection{Polynom dritten Grades}
Eine Funktion dritten Grades nennt man auch kubische Funktion.
Ihr Funktionsgraph sieht eher s-f\"ormig aus.
\[
	f(x) = a \cdot x^3 + b \cdot x^2 + c \cdot x + d ;
\]

\subsubsection{Polynom vierten Grades}
Der Graph einer Funktion vierten Grades kann w-f\"ormig aussehen.
\[
	f(x) = a \cdot x^4  + b \cdot x^3 + c \cdot x^2 + d \cdot x + e ;
\]


\subsection{Kurvendiskussion}
Zu einer vollst\"andigen Kurvendiskussion werden in der Regel folgende Punkte bestimmt:
\begin{itemize}[nosep]
	\item Definitionsmenge bzw. Grundmenge der Funktion
	\item Nullstelle(n), also Schnittpunkt(e) mit der $x$-Achse
	\item Schnittpunk(e) mit der $y$-Achse, also $ y = f(0) ; $
	\item Symmetrie(n)
	\item Grenzverhalten
	\item Extrempunkt(e)
	\item Wendepunkt(e)
	\item Funktionsgraph
	\item Monotonie-Verhalten
	\item Kr\"ummung
	\item Wertemenge
\end{itemize}

\subsubsection{Definitionsmenge bzw. Grundmenge der Funktion}
Um die Definitionsmenge zu bestimmen,
schaut man sich den Funktionsterm einfach an.
Bei einem allgemeinem Polynom ist die Definitionsmenge f\"ur gew\"ohnlich $ \mathbb{R} $.
Hat man jedoch bspw. Wurzeln oder Br\"uche,
ist es wichtig darauf zu achten,
welche Werte eingesetzt werden d\"urfen.
Der Radikand
(also das,
was unter der Wurzel steht)
darf nie negativ sein,
und bei einem Bruch darf der Nenner nicht Null sein.
Auch f\"ur andere Funktionen gelten bestimmte Werte.
Beispielsweise ist der
(nat\"urliche)
Logarithmus nur f\"ur positive Zahlen bestimmt.
Hier ein paar Beispiele:
\begin{itemize}[nosep]
	\item $ f(x) = \sqrt{x} ; \\ D = \mathbb{R}^{+}_{0}  $
	\item $ f(x) = \frac{1}{x} ; \\ D = \mathbb{R} \textbackslash \{ 0 \} $
	\item $ f(x) = \ln{x} ; \\ D = \mathbb{R}^{+} $
	\item $ f(x) = \sqrt{1 - x} ; \\ D = ] -\infty ; +1 ] $
\end{itemize}

\subsubsection{Symmetrie(n)}
Eine Symmetrie kann man bei einem Polynom relativ einfach bestimmen:
\begin{itemize}[nosep]
	\item Gibt es nur Polynome mit geraden Exponenten, ist der Funktionsgraph achsensymmetrisch.
	\item Gibt es nur Polynome mit ungeraden Exponenten, ist der Funktionsgraph punktsymmetrisch.
	\item Bei gemischten Polynomen gibt es h\"ochstwahrscheinlich keine Symmetrie.
\end{itemize}

\newpage
\section{Ableitung \& Stammfunktion}

\subsection{Allgemeine Ableitung}

Gegeben sei eine Funktion der allgemeinen Form:
\[
	 f(x) = a \cdot x^n ; 
\]

\subsubsection{Die erste Ableitung}
\[
	\frac{d}{dx} \left( a \cdot x^n \right) = f^\prime(x) = a \cdot n \cdot x^{n - 1} ;
\]

\subsubsection{Die zweite Ableitung}
\[
	f^{\prime\prime}(x) = a \cdot n \cdot \left( n - 1 \right) \cdot x^{n - 2} ;
\]

\subsubsection{Die dritte Ableitung}
\[
	f^{\prime\prime\prime}(x) = a \cdot n \cdot \left( n - 1 \right) \cdot \left( n - 2 \right) \cdot x^{n-3} ;
\]

\subsubsection{Die vierte Ableitung}
\[
	f^{(4)}(x) = a \cdot n \cdot \left( n - 1 \right) \cdot \left( n - 2 \right) \cdot \left( n - 3 \right) \cdot x^{n - 4} ;
\]

\subsection{Sonderf\"alle}

\subsubsection{Ableitung einer Konstanten}
Eine Konstante geht beim Ableiten immer verloren, weil sie $0$ wird.
\[
	\frac{d}{dx} \left( c \right) = 0 ;
\]

\subsubsection{Ableitung der Quadratwurzel}
\[
	\frac{d}{dx} \left( \sqrt{x} \right) = \frac{1}{2 \cdot \sqrt{x}} = \frac{\sqrt{x}}{2 \cdot x} ;
\]

\subsubsection{Ableitung der $n$-ten Wurzel}
\[
	\frac{d}{dx} \left( \sqrt[n]{x} \right) = \frac{1}{n} \cdot x^{\frac{1}{n} - 1} = \frac{1}{n \cdot \sqrt[n]{x^{n - 1}}} ;
\]

\subsubsection{Ableitung der Exponentialfunktion}
Die Exponentialfunktion leitet in ihrer Grundform immer auf sich selbst ab.
\[
	\frac{d}{dx} \left( e^x \right) = e^x ;
\]

\subsubsection{Ableitung des Logarithmus}
\[
	\frac{d}{dx} \left( ln(x) \right) = \frac{1}{x} ;
\]

\subsubsection{Ableitungen der trigonometrischen Funktionen}
Der Sinus und der Kosinus leiten im Prinzip immer wieder aufeinander ab:
\begin{align*}
	& \frac{d}{dx} \sin{x} = \cos{x} ;
	\\
	& \frac{d}{dx} \cos{x} = -\sin{x} ;
	\\
	& \frac{d}{dx} \left( -\sin{x} \right) = -\frac{d}{dx} \sin{x} = -\cos{x} ;
	\\
	& \frac{d}{dx} \left( -\cos{x} \right) = -\frac{d}{dx} cos{x} = \sin{x} ;
\end{align*}

Das hei{\ss}t,
die vierte Ableitung ist wieder die Originalfunktion.\\
Oder allgemein:
\[
	f^{(4n + k)} = f^{(k)} ;
\]

Hierbei steht $n$ f\"ur die $n$-te Ableitung, und $k$ steht f\"ur:
\begin{align*}
	& k = 0: \sin{x}
	\\
	& k = 1: \cos{x}
	\\
	& k = 2: -\sin{x}
	\\
	& k = 3: -\cos{x}
\end{align*}

\subsection{Weitere Ableitungsregeln}

\subsubsection{Summenregel}
Beim Ableiten einer Summe, kann jeder einzelne Summand f\"ur sich abgeleitet werden:
\[
	\frac{d}{dx} \left( f(x) + g(x) \right) = f^\prime(x) + g^\prime(x) ;
\]

\subsubsection{Faktorregel}
Die Faktorregel gilt f\"ur allgemeine Ableitung:
\[
	\frac{d}{dx} \left( a \cdot f(x) \right) = a \cdot f^\prime(x);
\]

\subsubsection{Potenzregel}
Die Potenzregel ist die allgemeine Ableitung:
\[
	\frac{d}{dx} \left( x^n \right) = n \cdot x^{n - 1} ;
\]

\subsubsection{Produktregel}
\[
	\frac{d}{dx} \left( f(x) \cdot g(x) \right) = f^\prime(x) \cdot g(x) + f(x) \cdot g^\prime(x) ;
\]

\subsubsection{Quotientenregel}
\[
	\frac{d}{dx} \left( \frac{f(x)}{g(x)} \right) = \frac{f^\prime(x) \cdot g(x) - f(x) \cdot g^\prime(x)}{ g^2(x) } ;
\]

\subsubsection{Kettenregel}
\[
	\frac{d}{dx} \left( f(g(x))  \right) = f^\prime(x) \cdot g^\prime(x) ;
\]
Hierbei muss bei der \"au{\ss}eren Ableitung das Argument unabgeleitet \"ubernommen werden.
\\ Beispiel:
\begin{align*}
	f(x) &= (x^4 + 5)^7 ;
	\\
	u &:= x^4 + 5 ;
	\\
	v &:= u^7 ;
	\\
	\\
	f^\prime(x) &= u^\prime \cdot v^\prime
	\\
	&= \frac{d}{dx} \left( x^4 + 5 \right) \cdot \frac{d}{dx} \left( u^7 \right)
	\\
	&= (4 \cdot x^3) \cdot (7 \cdot u^6)
	\\
	&= (4 \cdot x^3) \cdot 7 \cdot (x^4 + 5)^6
	\\
	&= 28 \cdot x^3 \cdot (x^4 + 5)^6 ;
\end{align*}

\subsection{Weitere Ableitungen}

\subsubsection{Trigonometrische Funktionen}
\begin{align*}
	& \frac{d}{dx} \tan{x} = \frac{d}{dx} \left( \frac{\sin{x}}{\cos{x}} \right) = 1 + \frac{\cos^2{x} + \sin^2{x}}{\cos^2{x}} = 1 + \tan^2{x} = \frac{1}{\cos^2{x}};
	\\
	& \frac{d}{dx} \cot{x} = \frac{d}{dx} \left( \frac{\cos{x}}{\sin{x}} \right) = -1 - \frac{\cos^2{x}}{\sin^2{x}} = -1 - \cot^2{x};
	\\
	& \frac{d}{dx} \sec{x} = \frac{d}{dx} \left( \frac{1}{\cos{x}} \right) = \frac{\sin{x}}{\cos^2{x}} = \frac{\tan{x}}{\cos{x}} = \sec{x} \cdot \tan{x} ;
	\\
	& \frac{d}{dx} \csc{x} = \frac{d}{dx} \left( \frac{1}{\sin{x}} \right) = \frac{-\cos{x}}{\sin^2{x}} = -\frac{\cot{x}}{\sin{x}} = -\cot{x} \cdot \csc{x} ;
	\\
	&
\end{align*}

\subsection{Allgemeine Stammfunktion}
Da beim Ableiten einer Funktion eine eventuelle Konstante verloren geht, spricht man in der Mathematik normalerweise nicht von \textbf{der}, sondern von \textbf{einer} Stammfunktion. Man spricht auch vom Integral einer Funktion.

\subsubsection{Das unbestimmte Integral}
\[
	F(x) = \int f(x) dx = \frac{a}{n + 1} \cdot x ^ {n + 1} + c ;
\]
Die Konstante $c$ (manchmal auch $C$) schreibt man aus formellen Gr\"unden mit.
Diese f\"allt ja bei der Ableitung heraus,
weil sie $0$ wird.
Desweiteren gilt:
\[
	\int -f(x) dx = -\int f(x) dx ;
\]


\subsubsection{Das bestimmte Integral}
\[
	\Bigl[ F(x) \Bigr]_{a}^{b} = \int_{a}^{b} f(x) dx = F(b) - F(a) ;
\]

\subsubsection{Das Integral von Sinus und Kosinus}
\"Ahnlich wie bei der Ableitung k\"onnen Sinus und Kosinus auch ineinander integriert werden:

Der Sinus und der Kosinus leiten im Prinzip immer wieder aufeinander ab:
\begin{align*}
	& \int \sin{x} dx = -\cos{x} ;
	\\
	& \int \cos{x} dx = \sin{x} ;
	\\
	& \int -\sin{x} dx = - \int \sin{x} dx = \cos{x} ;
	\\
	& \int -\cos{x} dx = - \int \cos{x} dx = -\sin{x} ;
\end{align*}

\subsubsection{Das Integral der Exponentialfunktion}
Da die Reinform der Exponentialfunktion auf sich selbst ableitet, kann sie auch mit sich selbst integriert werden:
\[
	\int e^x dx = e^x ;
\]

\subsection{Integrationsregeln}

\subsubsection{Partielle Integration (allgemein)}
Beim Integrieren gibt es keine allgemeing\"ultige Formeln, um das Integral komplett
aufzul\"osen. Es gibt jedoch ein paar Herangehensweisen, um die Integration zu erleichtern, und es nach und nach auszurechnen. Folgendes bietet sich an:
\begin{itemize}[nosep]
	\item Polynome werden differenziert \textit{(abgeleitet)}.
	\item Die Sinus- und Kosinusfunktionen k\"onnen in sich selbst integriert werden.
	\item Die Exponentialfunktion $ e^x $ kann auch in sich selbst integriert werden.
\end{itemize}

\subsubsection{Partielle Integration einer Summe}
\[
	\int \left( f(x) + g(x) \right) dx = \int f(x) dx + \int g(x) dx = F(x) + G(x) ;
\]

\subsubsection{Partielle Integration eines Produkts}
\[
	\int \left( f(x) \cdot g(x) \right) dx = F(x) \cdot g(x) - \int \left( F(x) \cdot g^\prime(x) \right) dx ;
\]
Da die Multiplikation kommutativ ist, gilt ebenso:
\[
	\int \left( f(x) \cdot g(x) \right) dx = f(x) \cdot G(x) - \int \left( f^\prime(x) \cdot G(x) \right) dx ;
\]
\textbf{Merkhilfe:} \textit{,,Man nimmt f\"ur eine der Funktionen eine Stammfunktion, und bildet das Produkt mit der anderen Funktion, minus das Integral von der Stammfunktion mal die Ableitung der anderen Funktion.'' (DorFuchs)}

\newpage
\section{Fakult\"at}

\subsection{Allgemein}
Die Fakult\"at einer Zahl $n \in \mathbb{N}_0$ ist im allgemeinen wie folgt definiert:
\[
	n! = n \cdot (n - 1) \cdot (n - 2) \cdot ... \cdot 3 \cdot 2 \cdot 1 ;
\]
\subsubsection{Sonderfall}
Die Fakult\"at von 0 ist mit 1 definiert:
\[
	0! = 1;
\]
Dies ergibt sich aus der logischen Folge:
\[
	n! = \frac{(n + 1)!}{n + 1} ;
\]
Also begonnen bei $5!$:
\begin{align*}
	& 5! = 5 \cdot 4 \cdot 3 \cdot 2 \cdot 1 = 120 ;
	\\
	& 4! = \frac{(4 + 1)!}{4 + 1} = \frac{5!}{5} = \frac{120}{5} = 24 ;
	\\
	& 3! = \frac{(3 + 1)!}{3 + 1} = \frac{4!}{4} = \frac{24}{4} = 6 ;
	\\
	& 2! = \frac{(2 + 1)!}{2 + 1} = \frac{3!}{3} = \frac{6}{3} = 2 ;
	\\
	& 1! = \frac{(1 + 1)!}{1 + 1} = \frac{2!}{2} = \frac{2}{2} = 1 ;
	\\
	& 0! = \frac{(0 + 1)!}{0 + 1} = \frac{1!}{1} = \frac{1}{1} = 1 ;
\end{align*}

\subsection{Berechnung \"uber die Ableitung}
Im Prinzip erh\"alt man die Fakult\"at \"uber die $n$-te Ableitung der folgenden Funktion:
\begin{align*}
	& f(x) = x^n ;
	\\
	& f^\prime(x) = n \cdot x^{n - 1} ;
	\\
	& f^{\prime\prime}(x) = n \cdot \left( n - 1 \right) \cdot x^{n - 2} ;
	\\
	& f^{\prime\prime\prime}(x) = n \cdot \left( n - 1 \right) \cdot \left( n - 2 \right)x^{n - 3} ;
	\\
	& ...
	\\
	& f^{(n)}(x) = n! ;
\end{align*}
Beispiel zum Berechnen der Fakult\"at von 5:
\begin{align*}
	& f(x) = x^5 ;
	\\
	& f^\prime(x) = 5 \cdot x^{5 - 1} = 5 \cdot x^4 ;
	\\
	& f^{\prime\prime}(x) = 5 \cdot 4 \cdot x^{4 - 1} = 20 \cdot x^3 ;
	\\
	& f^{\prime\prime\prime}(x) = 20 \cdot 3 \cdot x^{3 - 1} = 60 \cdot x^2;
	\\
	& f^{(4)}(x) = 60 \cdot 2 \cdot x^{2 - 1} = 120 \cdot x^1 = 120 \cdot x ;
	\\
	& f^{(5)}(x) = 120 \cdot 1 \cdot x^{1 - 1} = 120 \cdot x^0 = 120 \cdot 1 = 120 = 5! ;
\end{align*}

\subsection{Stirling-Formel}
Als Approximation \textit{(Ann\"aherung)} der Fakult\"at gibt es die sogenannte Stirling-Formel.
Je h\"oher hierbei $n$ ist,
desto geringer ist dabei der relative Fehler.
\[
	n! \approx \sqrt{2 \cdot \pi \cdot n} \cdot \left( \frac{n}{e} \right) ^n ;
\]

\subsubsection{Ableitung}
Theoretisch k\"onnte man die Stirling-Formel auch als eine beliebige,
reelle Funktion betrachten,
um auch z.b. die Gamma-Funktion,
die die Fakult\"at auf die reellen Zahlen erweitert,
zu approximieren:
\[
	x! \approx \Gamma(x + 1) \approx \sqrt{2 \cdot \pi \cdot x} \cdot \left( \frac{x}{e} \right) ^x ;
\]
Die Ableitung dieser Funktion ist sehr aufwendig, deswegen hier die Kurzfassung:
\[
	\frac{d}{dx} \left( \sqrt{2 \cdot \pi \cdot x} \cdot \left( \frac{x}{e} \right) ^x \right) = \left( \frac{x}{e} \right) ^x \cdot \left( \frac{\pi}{\sqrt{2 \cdot \pi \cdot x}} + \ln{x} \cdot \sqrt{2 \cdot \pi \cdot x} \right)
\]

\section{Trigonometrische Funktionen}

\subsection{Grundfunktionen}

\subsubsection{Sinus und Kosinus}
\begin{align*}
	&\sin{\alpha} = \frac{Gegenkathete}{Hypotenuse} ;
	\\
	&\cos{\alpha} = \frac{Ankathete}{Hypotenuse} ;
\end{align*}
Weiterhin gilt dass $ \sin^2{x} + \cos^2{x} = 1 $ ist,
weil:
\begin{align*}
	& \sin^2{x} + \cos^2{x} = \\
	& = \left( \frac{Gegenkathete}{Hypotenuse} \right) ^2 + \left( \frac{Ankathete}{Hypotenuse} \right) ^ 2 = \\
	& = \frac{Gegenkathete^2}{Hypotenuse^2} + \frac{Ankathete^2}{Hypotenuse^2} = \\
	& = \frac{Gegenkathete^2 + Ankathete^2}{Hypotenuse^2} ;
\end{align*}
Und weil im rechtwinkeligen Dreieck der \textbf{Satz des Pythagoras} gilt:
\[
Ankathete^2 + Gegenkathete^2 = Hypotenuse^2 ;
\]
kommt f\"ur $ \sin^2{x} + \cos^2{x} = 1 $ heraus.

\subsubsection{Tangens und Kotangens}
\begin{align*}
	&\tan{\alpha} = \frac{Gegenkathete}{Ankathete} = \frac{\sin{x}}{\cos{x}} ;
	\\
	&\cot{\alpha} = \frac{Ankatheta}{Gegenkathete} = \frac{\cos{x}}{\sin{x}} = \frac{1}{\tan{\alpha}} ;
\end{align*}

\subsubsection{Sekans und Kosekans}
\begin{align*}
	&\sec{\alpha} = \frac{Hypotenuse}{Ankathete} = \frac{1}{\cos{\alpha}} ;
	\\
	&\csc{\alpha} = \frac{Hypotenuse}{Gegenkathete} = \frac{1}{\sin{\alpha}} ;
\end{align*}

\subsection{Arkusfunktionen}
Die Arkusfunktionen sind die Umkehrfunktionen der trigonometrischen Funktionen.
F\"ur diese gibt es drei Schreibweisen.
Zum Beispiel kann der Arkussinus mit $ \arcsin $,
\textit{asin} oder $ sin^{-1} $ geschrieben werden.
Um Verwechslungen mit dem Kehrwert zu vermeiden,
verwende ich die Schreibweise mit \textit{arc} davor.

\subsubsection{Arkussinus und Arkuskosinus}

\newpage
\section{Zahlennamen}
Die Zahlen beziehen sich alle auf das Dezimalsystem.
Bei gro{\ss}en Zahlen gilt \"ubrigens,
wie im Deutschen \"ublich,
die gro{\ss}e Leiter,
d.h. tausend Millionen sind also eine Milliarde.

\subsection{Ziffern}
\begin{itemize}[nosep]
	\item 0: Null
	\item 1: Eins
	\item 2: Zwei
	\item 3: Drei
	\item 4: Vier
	\item 5: F\"unf
	\item 6: Sechs
	\item 7: Sieben
	\item 8: Acht
	\item 9: Neun
\end{itemize}

\end{document}
