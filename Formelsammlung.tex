\documentclass[a4paper]{article}

\usepackage{amsmath}
\usepackage{amssymb}
\usepackage{amsthm}
\usepackage[ngerman]{babel}
\usepackage[utf8]{inputenc}
\usepackage[T1]{fontenc}
\usepackage[left=25mm,right=25mm,top=25mm,bottom=25mm]{geometry}
\usepackage[colorlinks=true,urlcolor=blue,linkcolor=black]{hyperref}

\begin{document}

\setlength{\parindent}{0mm}

\title{Formelsammlung}
\author{Der Robert}
\maketitle

\tableofcontents
\newpage

\section{Ableitung \& Stammfunktion}

\subsection{Allgemeine Ableitung}

Gegeben sei eine Funktion der allgemeinen Form:
\[
	 f(x) = a \cdot x^n ; 
\]

\subsubsection{Die erste Ableitung}
\[
	\frac{d}{dx} \left( a \cdot x^n \right) = f^\prime(x) = a \cdot n \cdot x^{n - 1} ;
\]

\subsubsection{Die zweite Ableitung}
\[
	f^{\prime\prime}(x) = a \cdot n \cdot \left( n - 1 \right) \cdot x^{n - 2} ;
\]

\subsubsection{Die dritte Ableitung}
\[
	f^{\prime\prime\prime}(x) = a \cdot n \cdot \left( n - 1 \right) \cdot \left( n - 2 \right) \cdot x^{n-3} ;
\]

\subsubsection{Die vierte Ableitung}
\[
	f^{(4)}(x) = a \cdot n \cdot \left( n - 1 \right) \cdot \left( n - 2 \right) \cdot \left( n - 3 \right) \cdot x^{n - 4} ;
\]

\subsection{Sonderfälle}

\subsubsection{Ableitung einer Konstanten}
Eine Konstante geht beim Ableiten immer verloren, weil sie $0$ wird.
\[
	\frac{d}{dx} \left( c \right) = 0 ;
\]

\subsubsection{Ableitung der Quadratwurzel}
\[
	\frac{d}{dx} \left( \sqrt{x} \right) = \frac{1}{2 \cdot \sqrt{x}} = \frac{\sqrt{x}}{2 \cdot x} ;
\]

\subsubsection{Ableitung der $n$-ten Wurzel}
\[
	\frac{d}{dx} \left( \sqrt[n]{x} \right) = \frac{1}{n} \cdot x^{\frac{1}{n} - 1} = \frac{1}{n \cdot \sqrt[n]{x^{n - 1}}} ;
\]

\subsubsection{Ableitung der Exponentialfunktion}
Die Exponentialfunktion leitet in ihrer Grundform immer auf sich selbst ab.
\[
	\frac{d}{dx} \left( e^x \right) = e^x ;
\]

\subsubsection{Ableitung des Logarithmus}
\[
	\frac{d}{dx} \left( ln(x) \right) = \frac{1}{x} ;
\]

\subsubsection{Ableitungen von Sinus und Kosinus}
Der Sinus und der Kosinus leiten im Prinzip immer wieder aufeinander ab:
\begin{align*}
	& \frac{d}{dx} \left( sin(x) \right) = cos(x) ;
	\\
	& \frac{d}{dx} \left( cos(x) \right) = -sin(x) ;
	\\
	& \frac{d}{dx} \left( -sin(x) \right) = -cos(x) ;
	\\
	& \frac{d}{dx} \left( -cos(x) \right) = sin(x) ;
\end{align*}

Das heißt, die vierte Ableitung ist wieder die Originalfunktion.\\
Oder allgemein:
\[
	f^{(4n + k)} = f^{(k)} ;
\]

Hierbei steht $n$ für die $n$-te Ableitung, und $k$ steht für:
\begin{align*}
	& k = 0: sin(x)
	\\
	& k = 1: cos(x)
	\\
	& k = 2: -sin(x)
	\\
	& k = 3: -cos(x)
\end{align*}

\subsection{Weitere Ableitungsregeln}

\subsubsection{Summenregel}
Beim Ableiten einer Summe, kann jeder einzelne Summand für sich abgeleitet werden:
\[
	\frac{d}{dx} \left( f(x) + g(x) \right) = f^\prime(x) + g^\prime(x) ;
\]

\subsubsection{Faktorregel}
Die Faktorregel gilt für allgemeine Ableitung:
\[
	\frac{d}{dx} \left( a \cdot f(x) \right) = a \cdot f^\prime(x);
\]

\subsubsection{Potenzregel}
Die Potenzregel ist die allgemeine Ableitung:
\[
	\frac{d}{dx} \left( x^n \right) = n \cdot x^{n - 1} ;
\]

\subsubsection{Produktregel}
\[
	\frac{d}{dx} \left( f(x) \cdot g(x) \right) = f^\prime(x) \cdot g(x) + f(x) \cdot g^\prime(x) ;
\]

\subsubsection{Quotientenregel}
\[
	\frac{d}{dx} \left( \frac{f(x)}{g(x)} \right) = \frac{f^\prime(x) \cdot g(x) - f(x) \cdot g^\prime(x)}{ g^2(x) } ;
\]

\subsubsection{Kettenregel}
\[
	\frac{d}{dx} \left( f(g(x))  \right) = f^\prime(x) \cdot g^\prime(x) ;
\]
Hierbei muss bei der äußeren Ableitung das Argument unabgeleitet übernommen werden.
\\ Beispiel:
\begin{align*}
	f(x) &= (x^4 + 5)^7 ;
	\\
	u &:= x^4 + 5 ;
	\\
	v &:= u^7 ;
	\\
	\\
	f^\prime(x) &= u^\prime \cdot v^\prime
	\\
	&= \frac{d}{dx} \left( x^4 + 5 \right) \cdot \frac{d}{dx} \left( u^7 \right)
	\\
	&= (4 \cdot x^3) \cdot (7 \cdot u^6)
	\\
	&= (4 \cdot x^3) \cdot 7 \cdot (x^4 + 5)^6
	\\
	&= 28 \cdot x^3 \cdot (x^4 + 5)^6 ;
\end{align*}

\subsection{Allgemeine Stammfunktion}
Da beim Ableiten einer Funktion eine eventuelle Konstante verloren geht, spricht man in der Mathematik normalerweise nicht von \textbf{der}, sondern von \textbf{einer} Stammfunktion. Man spricht auch vom Integral einer Funktion.

\subsubsection{Das unbestimmte Integral}
\[
	F(x) = \int f(x) dx = \frac{a}{n + 1} \cdot x ^ {n + 1} + c;
\]
Die Konstante $c$ (manchmal auch $C$) schreibt man aus formellen Gründen mit. Diese fällt ja bei der Ableitung heraus, weil sie $0$ wird.

\subsubsection{Das bestimmte Integral}
\[
	\Bigl[ F(x) \Bigr]_{a}^{b} = \int_{a}^{b} f(x) dx = F(b) - F(a) ;
\]

\subsubsection{Das Integral von Sinus und Kosinus}
Ähnlich wie bei der Ableitung können Sinus und Kosinus auch ineinander integriert werden:

Der Sinus und der Kosinus leiten im Prinzip immer wieder aufeinander ab:
\begin{align*}
	& \int sin(x) dx = -cos(x) ;
	\\
	& \int cos(x) dx = sin(x) ;
	\\
	& \int -sin(x) dx = - \int sin(x) dx = cos(x) ;
	\\
	& \int -cos(x) dx = - \int cos(x) dx = -sin(x) ;
\end{align*}

\subsubsection{Das Integral der Exponentialfunktion}
Da die Reinform der Exponentialfunktion auf sich selbst ableitet, kann sie auch mit sich selbst integriert werden:
\[
	\int e^x dx = e^x ;
\]

\subsection{Integrationsregeln}

\subsubsection{Partielle Integration (allgemein)}
Beim Integrieren gibt es keine allgemeingültige Formeln, um das Integral komplett
aufzulösen. Es gibt jedoch ein paar Herangehensweisen, um die Integration zu erleichtern, und es nach und nach auszurechnen. Folgendes bietet sich an:
\begin{itemize}
	\item Polynome werden differenziert \textit{(abgeleitet)}.
	\item Die Sinus- und Kosinusfunktionen können in sich selbst integriert werden.
	\item Die Exponentialfunktion $ e^x $ kann auch in sich selbst integriert werden.
\end{itemize}

\subsubsection{Partielle Integration einer Summe}
\[
	\int \left( f(x) + g(x) \right) dx = \int f(x) dx + \int g(x) dx = F(x) + G(x) ;
\]

\subsubsection{Partielle Integration eines Produkts}
\[
	\int \left( f(x) \cdot g(x) \right) dx = F(x) \cdot g(x) - \int \left( F(x) \cdot g^\prime(x) \right) dx ;
\]
Da die Multiplikation kommutativ ist, gilt ebenso:
\[
	\int \left( f(x) \cdot g(x) \right) dx = f(x) \cdot G(x) - \int \left( f^\prime(x) \cdot G(x) \right) dx ;
\]

\section{Fakultät}

\subsection{Allgemein}
Die Fakultät einer Zahl $n \in \mathbb{N}_0$ ist im allgemeinen wie folgt definiert:
\[
	n! = n \cdot (n - 1) \cdot (n - 2) \cdot ... \cdot 3 \cdot 2 \cdot 1 ;
\]
\subsubsection{Sonderfall}
Die Fakultät von 0 ist mit 1 definiert:
\[
	0! = 1;
\]
Dies ergibt sich aus der logischen Folge:
\[
	n! = \frac{(n + 1)!}{n + 1} ;
\]
Also begonnen bei $5!$:
\begin{align*}
	& 5! = 5 \cdot 4 \cdot 3 \cdot 2 \cdot 1 = 120 ;
	\\
	& 4! = \frac{(4 + 1)!}{4 + 1} = \frac{5!}{5} = \frac{120}{5} = 24 ;
	\\
	& 3! = \frac{(3 + 1)!}{3 + 1} = \frac{4!}{4} = \frac{24}{4} = 6 ;
	\\
	& 2! = \frac{(2 + 1)!}{2 + 1} = \frac{3!}{3} = \frac{6}{3} = 2 ;
	\\
	& 1! = \frac{(1 + 1)!}{1 + 1} = \frac{2!}{2} = \frac{2}{2} = 1 ;
	\\
	& 0! = \frac{(0 + 1)!}{0 + 1} = \frac{1!}{1} = \frac{1}{1} = 1 ;
\end{align*}

\subsection{Berechnung über die Ableitung}
Im Prinzip erhält man die Fakultät über die $n$-te Ableitung der folgenden Funktion:
\begin{align*}
	& f(x) = x^n ;
	\\
	& f^\prime(x) = n \cdot x^{n - 1} ;
	\\
	& f^{\prime\prime}(x) = n \cdot \left( n - 1 \right) \cdot x^{n - 2} ;
	\\
	& f^{\prime\prime\prime}(x) = n \cdot \left( n - 1 \right) \cdot \left( n - 2 \right)x^{n - 3} ;
	\\
	& ...
	\\
	& f^{(n)}(x) = n! ;
\end{align*}
Beispiel zum Berechnen der Fakultät von 5:
\begin{align*}
	& f(x) = x^5 ;
	\\
	& f^\prime(x) = 5 \cdot x^{5 - 1} = 5 \cdot x^4 ;
	\\
	& f^{\prime\prime}(x) = 5 \cdot 4 \cdot x^{4 - 1} = 20 \cdot x^3 ;
	\\
	& f^{\prime\prime\prime}(x) = 20 \cdot 3 \cdot x^{3 - 1} = 60 \cdot x^2;
	\\
	& f^{(4)}(x) = 60 \cdot 2 \cdot x^{2 - 1} = 120 \cdot x^1 = 120 \cdot x ;
	\\
	& f^{(5)}(x) = 120 \cdot 1 \cdot x^{1 - 1} = 120 \cdot x^0 = 120 \cdot 1 = 120 = 5! ;
\end{align*}

\subsection{Stirling-Formel}
Als Approximation \textit{(Annäherung)} der Fakultät gibt es die sogenannte Stirling-Formel. Je höher hierbei $n$ ist, desto geringer ist dabei der relative Fehler.
\[
	n! \approx \sqrt{2 \cdot \pi \cdot n} \cdot \left( \frac{n}{e} \right) ^n ;
\]

\end{document}
